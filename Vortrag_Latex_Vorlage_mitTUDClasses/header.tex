% \usepackage{multimedia}
% \usepackage[square]{natbib}
% \usepackage{bibgerm}
\usepackage{tikz}

\usetikzlibrary{shapes.geometric,shapes.arrows,decorations.pathmorphing}
\usetikzlibrary{matrix,chains,scopes,positioning,arrows,fit}
\usetikzlibrary{calc}
\usetikzlibrary{patterns}

% \usetikzlibrary{patterns}
 \usetikzlibrary{matrix}
 \usetikzlibrary{decorations.pathmorphing}
 \usetikzlibrary{arrows,shapes.geometric} % fadings, ???
 \usepackage{fancybox}
 \usepackage{amsmath}            % Mathematik-Umgebungen
\usepackage{bm}                 % fette Mathe Buchstaben
\usepackage{amssymb}            % spezielle Mathe-Symbole


\usepackage{color}

%
% Absolute positioning
% beamer paper size: 12.80cm x 9.60cm
%
\usepackage[overlay,absolute]{textpos}
\setlength{\TPHorizModule}{10mm}
\setlength{\TPVertModule}{\TPHorizModule}
\textblockorigin{5mm}{25.5mm}
\setlength{\parindent}{0pt}

\newcommand{\leftbox}[1]{
 \begin{textblock}{5.7}(0, 0)
   #1
 \end{textblock}
}

\newcommand{\rightbox}[1]{
 \begin{textblock}{5.7}(6.1, 0)
   #1
 \end{textblock}
}

\newcommand{\fullbox}[1]{
 \begin{textblock}{11.8}(0, 0)
   #1
 \end{textblock}
}

% \usepackage{calc}
\newenvironment{arbbox}[2]
{%\setlength{\boxwidth}{11.8cm minus \real{#1}}
\begin{textblock}{11.8cm}(#1, #2)}
{\end{textblock}}


