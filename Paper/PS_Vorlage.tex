\documentclass[journal,final,a4paper,twoside]{PS}

%%% Dieser Block ist dem Betreuer des Projektseminars vorbehalten
\usepackage{PS}             % Alle Definitionen �ber den Seitenstil (auf keinen Fall editieren!!)
\usepackage[T1]{fontenc}
\usepackage[latin1]{inputenc}

\def\lehrveranstaltung{PROJEKTSEMINAR ROBOTIK UND COMPUTATIONAL INTELLIGENCE}
\def\ausgabe{Vol.17,~SS~2017}
\setcounter{page}{1}        % Hier die Seitennummer der Startseite f�r Gesamtdokument festlegen

%%% Ab hier k�nnen Eintr�ge von den Teilnehmern des Projektseminars gemacht werden
%%% Wenn neben den LaTeX-Paketen aus der Datei PS.sty noch weitere gebraucht werden,
%%% so ist dies dringend mit dem Betreuer abzukl�ren!

\begin{document}
\newcommand{\euertitel}{Dynamic Distortion Calibration}   % Titel hier eintragen!
\newcommand{\betreuer}{M. Sc. Raul Acuna }  % Betreuerdaten hier eintragen (mit einem Leerzeichen am Ende)!

\headsep 40pt
\title{\euertitel}
% Autorennamen in der Form "Vorname Nachname" angeben, alphabetisch nach Nachname sortieren,
% nach dem letzen Autor kein Komma setzen, sondern mit \thanks abschlie�en
\author{Ahmed Ashraf,
        Nils Hamacher,
        Linghan Qian,
	Vivica Wirth
\thanks{Diese Arbeit wurde von \betreuer unterst�tzt.}}

\maketitle


%\begin{Zusammenfassung}
%german translation necessry?
%\end{Zusammenfassung}
%\vspace{6pt}

\begin{abstract}

All camera lenses are somewhat different from each other. The curvature of a lens is never perfect, and inclusions may occur which also alter the curvature of the incident light. This leads to distortion that has to be calibrated. State of the art approaches like the checkerboard calibration only base their models on a few interest points. Our aim was to come up a program, which automatically creates a dense model of the lens distortion. This dense model is based on information of the correspondencies of pixels on a screen and the pixels in a captured image. Once the mapping between the screen and the image corresponding pixels is done, we can undistort the image simply by moving the pixels of the actual image we take. In order to reach the desired accuracy we need to set some pre-conditions for the environment in which this takes place. These conditions are darkness, in order to avoid reflections of the camera on the screen, and the camera being perpendicular to the screen. We will present the fundamental principles of several different ideas, which all result in a mapping. To compare them we give a short insight to their runtimes.
\end{abstract}

\section{Introduction}

\PARstart{C}{ameras} are used in more and more environments. They are used as surveillance technology, in toys, mobile phones, and also increasingly in vehicles. 
The demand for cameras with improving resolution is also increasing. For this purpose, a method for automatic calibration with high accuracy that does not require human interaction is an advantage. At present, the most common method for calibration is to position a checkerboard in different poses in front of the camera and detect the intersection of the black and white squares. 



Bitte beachtet auch die Hinweise zum Verfassen wissenschaftlicher Texte in Anhang~\ref{sec:richtlinien} und~\ref{sec:notation}.

In der Einf�hrung sollte kurz beschrieben werden, was die Problemstellung der Arbeit ist, welche Vorarbeiten es schon gibt (``Stand der Technik'' mit Verweis auf passende Quellen) und was der neue Beitrag der Arbeit ist. Am Ende der Einf�hrung kann kurz auf die Struktur des Artikels eingegangen werden, z.B.: 

Abschnitt~\ref{sec:grundlangen} f�hrt wichtige Grundlagen ein und Abschnitt~\ref{sec:zus} fasst schlie�lich die Ergebnisse zusammen und gibt einen Ausblick.

\section{Grundlagen}
\label{sec:grundlangen}

\subsection{Dies ist ein Unterabschnitt}
Subsection text.
\subsubsection{Dies ist ein Unter-Unterabschnitt}
Subsubsection text.




\section{Zusammenfassung}
\label{sec:zus}
Hier die wichtigsten Ergebnisse der Arbeit in 5-10 S�tzen zusammenfassen. Dies sollte keine Wiederholung des Abstracts oder der Einf�hrung sein. Insbesondere kann hier ein Ausblick auf zuk�nftige Arbeiten gegeben werden.


\appendices
\section{Optionaler Titel}
Anhang eins.
\section{}
Anhang zwei.

\section{Richtlinien f�r das Verfassen wissenschaftlicher Arbeiten}
\label{sec:richtlinien}
Im Folgenden werden einige wichtige Richtlinien zusammengefasst. Die Aufzaehlung ist allerdings nicht ersch�pfend.

\begin{itemize}
 \item Klare Darstellung, was der Eigenanteil ist und was schon vorhanden war.
 \item Vorsicht vor Plagiaten: vollst�ndige Quellenangaben, auch bei Bildern. Es sollte immer klar ersichtlich sein, was der Eigenanteil ist und was aus Quellen entnommen wurde.
 \item Bilder nicht 1:1 aus Quellen kopieren.
 \item Diskussion der Ergebnisse (Simulationen, Messungen, Rechnungen): Wurde das Ergebnis so erwartet? Wenn nein, was sind m�gliche Gr�nde?
 \item Autoren: Als Autor sollte jede Person in Betracht gezogen werden, die wesentlich zur Arbeit beigetragen hat (siehe auch die Empfehlungen der DFG diesbez�glich, vgl.~\cite{wissPraxis:DFG}). Alle Personen mit kleinerem Beitrag (fachliche Hinweise, Beteiligung an Datensammlung etc.) k�nnen in der Danksagung oder einer Fu�note erw�hnt werden.
\item Formeln in den Satz einbetten und alle Variablen bei der ersten Verwendung im Text einf�hren. Beispiel:
 F�r die Temperatur ergibt sich damit
 \begin{align*}
  T(h) = K h^2,
 \end{align*}
 sie h�ngt quadratisch von der H�he $h$ ab.
\end{itemize}

\section{Hinweise zur Notation}
\label{sec:notation}
\begin{itemize}
 \item Abk�rzungen bei der ersten Verwendung erkl�ren, z.B.: ``DFG (Deutsche Forschungsgemeinschaft)''.
 \item Formelzeichen konsistent benennen, nicht zwischen den Abschnitten umbenennen. Formelzeichen kursiv schreiben, z.B. Variable $a$.
 \item Auf korrekte Dimensionen und Einheiten achten. F�r Einheiten das SI-System verwenden, z.B. das LaTeX-Paket \emph{units} oder \emph{SIunits}.
 \item Zahlen: Im Deutschen Komma als Dezimaltrennzeichen, im Englischen Punkt.
 \item Tabellen haben �berschriften, Diagramme haben Unterschriften.
 \item Diagramme: Achsenbeschriftungen hinreichend gro� (insbesondere die Zahlen).
 \item Diagrammunterschriften sollen im Wesentlichen ausreichen, um das Diagramm zu verstehen.
 \item Indizes werden \emph{kursiv} gesetzt, wenn sie die Bedeutung von Variablen haben, ansonsten \textbf{normal}. Beispiele: $V_k, \ k=1,2,\ldots$ und $V_\mathrm{input}$.
\end{itemize}




\section*{Danksagung}
Wenn ihr jemanden danken wollt, der Euch bei der Arbeit besonders
unterst�tzt hat (Korrekturlesen, fachliche Hinweise,...), dann ist hier der daf�r vorgesehene Platz.

\begin{thebibliography}{1}
\bibitem{IEEEhowto:kopka}
H.~Kopka and P.~W. Daly, \emph{A Guide to {\LaTeX}}, 3rd~ed. Harlow, England: Addison-Wesley, 1999.
\bibitem{wissPraxis:DFG}
Deutsche Forschungsgemeinschaft, \emph{Vorschl�ge zur Sicherung guter wissenschaftlicher Praxis}, Denkschrift, Weinheim: Wiley-VCH, 1998.
\end{thebibliography}

\begin{biography}
[{\includegraphics[width=1in,height=1.25in,clip,keepaspectratio]{./pics/ComicKopf.eps}}] % hier ein Foto einbinden
{Autor A}
Biographie Autor A.
\end{biography}
\begin{biography}
[{\includegraphics[width=1in,height=1.25in,clip,keepaspectratio]{./pics/ComicKopf.eps}}] % hier ein Foto einbinden
{Autor B}
Biographie Autor B.
\end{biography}
\begin{biography}
[{\includegraphics[width=1in,height=1.25in,clip,keepaspectratio]{./pics/ComicKopf.eps}}] % hier ein Foto einbinden
{Autor C}
Biographie Autor C.
\end{biography}

\end{document}